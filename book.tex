\documentclass[11pt]{report}
\usepackage[utf8]{inputenc}
\usepackage{adjustbox}
\usepackage{graphicx}
\usepackage{amsmath,amssymb,latexsym}
\usepackage{mathtools}
\DeclarePairedDelimiter\floor{\lfloor}{\rfloor}
\DeclarePairedDelimiter\norm{\lVert}{\rVert}
\DeclarePairedDelimiter\abs{\lvert}{\rvert}%

\usepackage{fullpage}
\usepackage{graphicx}
\let\phi\varphi
\newcommand{\Z}{\mathbb{Z}}
\newcommand{\R}{\mathbb{R}}
\newcommand{\F}{\mathbb{F}}
\newcommand{\C}{\mathbb{C}}
\newcommand{\Q}{\mathbb{Q}}
\newcommand{\N}{\mathbb{N}}

\usepackage{fancyhdr}
\setlength{\headheight}{14pt} 
\pagestyle{fancy}
\lhead{\nouppercase{\rightmark} (\nouppercase{\leftmark})}
\chead{}
\rhead{}
\lfoot{\today}
\cfoot{}
\rfoot{\thepage}
\renewcommand{\headrulewidth}{0.4pt}
\renewcommand{\footrulewidth}{0.4pt}

 \renewcommand{\chaptermark}[1]{%
 \markboth{#1}{}}

\begin{document}
\title{Linear Algebra}
\author{Dilip Thiagarajan}
\maketitle
\tableofcontents{}

\chapter{Groups, Rings and Fields}
The concept of mathematical sets is important in all fields of mathematics, and is especially true in linear algebra given the ubiquity of vector spaces and subspaces. As such, the idea of fields, a close analog to vector spaces, is quite important to mention in a proper reading of the topic, and thus, we begin by introducing groups, upon which we build rings, which we consequently use to build a field. 
\section{Groups}
\section{Rings}
\section{Fields}


\chapter{Vector Spaces, Subspaces and Quotient Spaces}


\chapter{Spans, Linear Independence and Bases}


\chapter{Linear Transformations and the Isomorphism Theorems}
\section{Nilpotent Transformations}
\section{Projection Transformations}
\chapter{Matrices and Linear Systems}


\chapter{Applications}
At this point, we bring up some interesting applications that require only the knowledge of solving linear systems using basic row reduction operations.
\section{Discrete Dynamics}
\section{Markov Chains}
\section{Stochastic Matrices}


\chapter{Determinants, Invertibility, and Eigen-theory}
In this chapter, we'll introduce the determinant function, which is a special function (in its alternating and mulitinear characteristic) that allows us to introduce another perspective of linear transformations. More specifically, we'll look at how transformations can be inverted (i.e. when they are bijective), and see how this may be useful in developing the idea of similar transformations.
\section{Determinants}
\section{Invertibility}
\section{Eigenvalues and Eigenvectors}
\section{Diagonalization and Similarity}
\section{Spectral Value Decomposition}


\chapter{Inner Products}


\chapter{Adjoints, Spectral Theorem, Principal Axis Theorem}


\chapter{Jordan and Rational Canonical Forms}
asdf
\section{Invariant Subspaces}
\section{Jordan Canonical Forms}
\section{Rational Canonical Forms}
\section{Applications}


\chapter{Application to Differential Equations}


\chapter{The Similarity Problem}

\end{document}